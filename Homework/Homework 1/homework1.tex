%%% Template originaly created by Karol Kozioł (mail@karol-koziol.net) and modified for ShareLaTeX use

\documentclass[11pt]{article}

\usepackage[T1]{fontenc}
\usepackage[utf8]{inputenc}
\usepackage{graphicx}
\usepackage{xcolor}
\usepackage{float}
\usepackage{tgtermes}
\usepackage{natbib}
%\usepackage[subnum]{cases}
\usepackage[super]{nth}
\bibpunct{(}{)}{;}{a}{}{,}
\usepackage{amsmath,amssymb}
\usepackage{enumerate}
\usepackage{multicol}
\usepackage{tikz}
\usepackage[amssymb]{SIunits}
\usepackage{rotating}
\usepackage{enumitem}
\usepackage{geometry}
\geometry{total={8.5in,11in},
left=1in,right=1in,%
bindingoffset=0mm, top=1in,bottom=1in}
\usepackage[super]{nth}
\usepackage[
pdftitle={EFD: Homework 1}, 
pdfauthor={Jeremy Gibbs, University of Utah},
colorlinks=true,linkcolor=blue,urlcolor=blue,citecolor=blue,bookmarks=true,
bookmarksopenlevel=2]{hyperref}

\linespread{1.1}
\setlength{\parskip}{1em}
\setlength{\parindent}{0pt}
\newcommand{\linia}{\rule{\linewidth}{0.5pt}}

\makeatletter
\renewcommand{\maketitle}{
\begin{center}
\vspace{2ex}
{\huge \textsc{\@title}}
\vspace{1ex}
\\
\linia\\
ME EN 7710 \hfill Homework \#1 \hfill Due: January \nth{24}
\vspace{4ex}
\end{center}
}
\makeatother
%%%

% custom footers and headers
\usepackage{fancyhdr,lastpage}
\pagestyle{fancy}
\lhead{}
\chead{}
\rhead{}
%\lfoot{Assignment \textnumero{} 5}
\cfoot{}
\rfoot{Page \thepage~/~\pageref*{LastPage}}
\renewcommand{\headrulewidth}{0pt}
\renewcommand{\footrulewidth}{0pt}
%

%%%----------%%%----------%%%----------%%%----------%%%

\begin{document}

\title{Environmental Fluid Dynamics}

\maketitle

%-- Question 1 --%
\vspace{-20pt}
\paragraph{1.) Arya - Chapter 2, Exercise 5}~\\\\
Explain the following terms or concepts used in connection with the surface energy budget:
\begin{enumerate}[label=(\alph*)]
	\item ``ideal'' surface
	\item evaporative cooling
	\item oasis effect
	\item flux divergence
\end{enumerate}
 
%-- Question 2 --%
\paragraph{2.) Arya - Chapter 3, Exercise 2}~\\
\begin{enumerate}[label=(\alph*),topsep=-10pt]
	\item Estimate the combined sensible and latent heat fluxes from the surface to the atmosphere, given the following observations:
\vspace{-10pt}
\begin{itemize}
	\item Incoming shortwave radiation = $800\ \watt\; \metre\rpsquared$
	\item Heat flux to the submedium = $150\ \watt\; \metre\rpsquared$
	\item Albedo of the surface = 0.35
\end{itemize}
	\item What would be the result if the surface albedo were to drop to 0.07 after irrigation?
\end{enumerate}

%-- Question 3 --%
\paragraph{3.) Arya - Chapter 3, Exercise 3}~\\\\
The following measurements or estimates were made of the radiative fluxes over a short grass surface during a clear sunny day:
\vspace{-10pt}
\begin{itemize}
	\item Incoming shortwave radiation: $675\ \watt\; \metre\rpsquared$
	\item Incoming longwave radiation: $390\ \watt\; \metre\rpsquared$
	\item Ground surface temperature: 35$^\circ$C
	\item Albedo of the surface: 0.20
	\item Emissivity of the surface: 0.92
\end{itemize}
\begin{enumerate}[label=(\alph*),topsep=-10pt]
	\item From the radiation balance equation, calculate the net radiation at the surface.
	\item What would be the net radiation after the surface is thoroughly watered so that its albedo drops to 0.10 and its effective surface temperature reduces to 25$^\circ$C?
	\item Qualitatively discuss the effect of watering on the other energy fluxes to or from the surface.
\end{enumerate}

%-- Question 4 --%
\paragraph{4.) Arya - Chapter 3, Exercise 7}~\\\\
Discuss the merits of the proposition that net radiation $R_N$ can be deduced from measurements from solar radiation $R_{S\downarrow}$ during the daylight hours, using the empirical relationship $$R_N = AR_{s\downarrow} + B$$ where $A$ and $B$ are constants. On what factors are $A$ and $B$ expected to depend?

%-- Question 5 --%
\paragraph{5.) Boltzmann and Planck}~\\\\
Derive Stefan-Boltzmann's Law from Planck's Law.

%-- Question 5 --%
\paragraph{6.) Wein and Planck}~\\\\
Derive Wein's Law from Planck's Law.
\end{document}