%%% Template originaly created by Karol Kozioł (mail@karol-koziol.net) and modified for ShareLaTeX use

\documentclass[11pt]{article}

\usepackage[T1]{fontenc}
\usepackage[utf8]{inputenc}
\usepackage{graphicx}
\usepackage{xcolor}
\usepackage{float}
\usepackage{tgtermes}
%\usepackage{natbib}
%\usepackage[subnum]{cases}
\usepackage[super]{nth}
%\bibpunct{(}{)}{;}{a}{}{,}
\usepackage{amsmath,amssymb}
\usepackage{enumerate}
\usepackage{multicol}
\usepackage{tikz}
\usepackage[amssymb]{SIunits}
\usepackage{rotating}
\usepackage{enumitem}
\usepackage{geometry}
\geometry{total={8.5in,11in},
left=1in,right=1in,%
bindingoffset=0mm, top=1in,bottom=1in}
\usepackage[super]{nth}
\usepackage[
pdftitle={Project 2}, 
pdfauthor={Jeremy Gibbs, University of Utah},
colorlinks=true,linkcolor=blue,urlcolor=blue,citecolor=blue,bookmarks=true,
bookmarksopenlevel=2]{hyperref}

\linespread{1.1}
\setlength{\parskip}{1em}
\setlength{\parindent}{0pt}
\newcommand{\linia}{\rule{\linewidth}{0.5pt}}

\makeatletter
\renewcommand{\maketitle}{
\begin{center}
\vspace{2ex}
{\huge \textsc{\@title}}
\vspace{1ex}
\\
\linia\\
ME EN 7710 \hfill Project \#2 \hfill Due May 1\\
Atmospheric Surface Layer Turbulence Analysis
\vspace{4ex}
\end{center}
}
\makeatother
%%%

% custom footers and headers
\usepackage{fancyhdr,lastpage}
\pagestyle{fancy}
\lhead{}
\chead{}
\rhead{}
%\lfoot{Assignment \textnumero{} 5}
\cfoot{}
\rfoot{Page \thepage~/~\pageref*{LastPage}}
\renewcommand{\headrulewidth}{0pt}
\renewcommand{\footrulewidth}{0pt}
%

%%%----------%%%----------%%%----------%%%----------%%%

\begin{document}
\bibliographystyle{abbrv}
\title{Environmental Fluid Dynamics}

\maketitle
\vspace{-20pt}
\paragraph{\large Overview}~\\~\\
You will investigate various aspects of turbulence using data from the MATERHORN field campaign conducted at Dugway Proving Ground. Information on the campaign can be found at: \href{http://www3.nd.edu/~dynamics/materhorn/}{http://www3.nd.edu/~dynamics/materhorn/}. Additional project resources are posted on Canvas and the \href{http://gibbs.science/efd}{course website}. Each group should obtain the sonic anemometer/thermometer data directly from me.

 The purpose of this project is to understand the physics of turbulent flow occurring at the measurement site. You will need to come up with a general scientific objective and hypothesis that you intend to test, present on, and write-up. For example, a group may choose as an objective: ``To better understand the effect of atmospheric stability on turbulence for a canyon outflow'' and a hypothesis that might be tested could be: ``Night-time turbulence is enhanced by the formation of surface level jet.''

You must first select a time period for analysis. You may select the entire day for simple statistics and tethered balloon analysis, but you will want to focus on a shorter time period for the turbulence analysis that is required and outlined below. You may work in groups of two or three. Please select a project topic as soon as possible and obtain my approval. 

Oral presentations: \textbf{Monday, May 1 from 10:30a-12:30p}. Final report due: \textbf{Monday, May 1 at 5:00p}.

\paragraph{\large Assignment}~\\~\\
Using the Sonic Anemometer data, please perform the following analysis:
\vspace{-10pt}
\begin{enumerate}
	\item \textbf{Simple Time Averaging:} calculate (a) 30-minute averages of $u$, $v$, $w$, and $T$, as well as (b) $\overline{ws}$, $\overline{wd}$, $\sigma_u$, $\sigma_v$, $\sigma_w$, $\overline{w^\prime T_s^\prime}$, $u_*$, $H_s$, $tke$, $L$, and $w_*$ (if appropriate), where $\overline{ws}$ and $\overline{wd}$ are average wind speed and direction, respectively. Describe the stability during the analyzed period.
	\item \textbf{Probability Distributions:} For a representative 30-minute averaging period, generate a CDF and PDF for $u$, $v$, $w$, and $T$ and report the skewness and kurtosis.
	\item \textbf{Autocorrelation:} Calculate the autocorrelation of at least one 30-minute period. What does this indicate?
	\item \textbf{Dissipation: } Using Taylor's frozen turbulence hypothesis, calculate the dissipation rate of turbulent kinetic energy for several 30-minute periods. Calculate the Kolmogorov length scale.
	\item \textbf{Turbulence Spectra:} For at least one 30-minute averaging period, calculate the following turbulent energy spectra: $S_{uu}$, $S_{vv}$, $S_{ww}$, $S_{TT}$, as well as the following cospectra: $S_{uw}$, $S_{vw}$, $S_{wt}$. What do these spectra indicate about the analyzed boundary layer? Is there an inertial subrange?
\end{enumerate}

\end{document}